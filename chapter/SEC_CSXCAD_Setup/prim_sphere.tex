Introduce a sphere into\hyperref[CSX]{\matv{CSX}}. 

\begin{FontNameFunct}{AddSphere()}
\end{FontNameFunct}


\begin{FontDescr}{Purpose:}
To add a sphere into\matv{CSX}\phantomsection\label{CSX} by defining its radius and center point and assign a material property to it.  
\end{FontDescr}

\begin{FontDescr}{Syntax:}
\begin{lstlisting} 
CSX = AddSphere(CSX, propName, prio, center, rad, varargin)
\end{lstlisting}
\end{FontDescr}

\begin{FontDescr}{Description:}

\begin{FontPara}{propName}
Refer to \hyperref[prim_Name]{propName} in \texttt{AddBox}. 
\end{FontPara}

\begin{FontPara}{center}
Sphere center point.(Vector) 
\end{FontPara}

\begin{FontPara}{rad}
Radius of a sphere.
\end{FontPara}
\end{FontDescr}

\begin{FontDescr}{Optional Arguments:}
The standard trasformation (rotation,translation,scaling) mentioned in  \hyperref[prim_transform]{\matv{'Transform'}} of \texttt{AddBox}.   
\end{FontDescr}

\begin{FontDescr}{Examples:}

\begin{lstlisting} 
CSX = AddMetal(CSX,'metal'); 
CSX = AddSphere(CSX,'metal',10,[0 0 0],50);
\end{lstlisting}
This example creates a metallic sphere of radius 50(drawing unit) at origin. 

\begin{lstlisting} 
CSX = AddMetal(CSX,'metal'); 
CSX = AddSphere(CSX,'metal1',10,[0 0 0],10,'Transform',{'Translate','0,0,50'});  
\end{lstlisting}
The above sphere has been shifted in z-direction 50(drawing unit) above the origin. 

\end{FontDescr}



