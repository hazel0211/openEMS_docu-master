This section shows how a metal/PEC material is introduced into
 \hyperref[CSX]{\matv{CSX}}.
 
\begin{FontNameFunct}{AddMetal}\phantomsection\label{addmetal}
\end{FontNameFunct}

\begin{FontDescr}{Syntax:}
  \begin{lstlisting}
 CSX = AddMetal(CSX, name)
  \end{lstlisting}
\end{FontDescr}  

\begin{FontDescr}{Description:}
This function introduces perfect electric conductor of no loss into \hyperref[CSX]{\matv{CSX}}.
\end{FontDescr}

 
\begin{FontNameFunct}{AddConductingSheet}
\end{FontNameFunct}

\begin{FontDescr}{Purpose:}
To add a lossy conducting material into \hyperref[CSX]{\matv{CSX}}.
\end{FontDescr}

\begin{FontDescr}{Syntax:}
  \begin{lstlisting}
CSX = AddConductingSheet(CSX, name, conductivity, thickness)
  \end{lstlisting}
\end{FontDescr} 

\begin{FontDescr}{Description:}

\begin{FontPara}{conductivity}
It is given by user according to the conductivity of metal being used. The most frequent used metal is copper and its conductivity is 58e6$Sm^{-1}$.    
\end{FontPara}

 \begin{FontPara}{thickness}
 It defines the thickness of lossy metal sheet.     
 \end{FontPara}
\end{FontDescr} 

\begin{FontDescr}{Examples:} 

\begin{lstlisting} 
CSX = AddMetal(CSX,'metal'); 
\end{lstlisting}
This syntax adds PEC material into \hyperref[CSX]{\matv{CSX}} with the name 'metal'. \\

\begin{lstlisting} 
CSX = AddConductingSheet(CSX,'copper',56e6,70e-6);
\end{lstlisting}
This syntax adds a conducting sheet of 70$\mu$m into \hyperref[CSX]{\matv{CSX}}. The assigned metal is named copper and has conductivity of 58e6$Sm^{-1}$.  

 \end{FontDescr} 
 
 
 
 